\documentclass{article}
\usepackage[utf8]{inputenc}

\title{Geographical embedding of learning agents}
\author{}
\date{}

\begin{document}

\maketitle

see https://distill.pub/2020/growing-ca/ ?

\section{Introduction}

Idea: embed the deep learning agents of Mathy https://mathy.ai/api/models/ into a geographical space 

\begin{itemize}
    \item Do coevolution niche emerge?
    \item Do different regions discover ``different ways to do maths''?
    \item optimal modularity for global/local performance; rate of knowledge exchange between communities?
\end{itemize}

Close to ideas of biogeography evolutionary algorithms?

\cite{mirjalili2014let} trains deep neural networks with a biogeography algorithm. What are exchanged / mutated are part of each neural networks - populations are neurons. Not exactly similar to populations of neural networks? No heuristic process to justify migration of some neurons only.


\cite{bruce2001evolving}

\cite{lund1995preadaptation} coevolution with the environment: importance of differentiated context?

\cite{cangelosi1998emergence} emergence of a language when categorizing elements in the environment give an evolutionary advantage

\cite{nolfi1994learning} agent-level learning/global evolutionary dynamics

\cite{beer1992evolving} GA evolving neural networks (feedforward in time)

\cite{stanley2002evolving}

$\rightarrow$ use local GA processes to select / mutate agents which are neural networks? The fitness is the score to solve the mathematical problem. Changing environment: different problems in different parts of the space? add movements by agents? (move away if cannot solve - exchange with neighborhood agents - tolerance to failure).

\cite{denaro1997cultural}


\end{document}
